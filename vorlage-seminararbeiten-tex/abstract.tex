\section*{Kurzfassung}
\thispagestyle{empty}


Aus Zeit- und Kostengründen beim Entwickeln und Testen von komplexen Systemen werden Tools zur Programmcodevalidierung immer relevanter.
Diese Tools ermöglichen das Schreiben von Programmen, welche mathematisch und maschinell geprüft sind. Dadurch ist sichergestellt, dass das beschriebene Programm sich auch wie gewünscht, verhält.\\
Ziel dieser Arbeit ist es, einen sowohl theoretischen als auch technischen Einblick in die Programmcodevalidierung mit dem Proof Assistent Tool Coq darzustellen.
Als Einstieg werden die grundlegende Begriffe geklärt und ein kurzer Überblick über Tools in diesem Fachbereich dargestellt. Dabei wird insbesondere auf Coq eingegangen.\\
Um ein Verständnis zu bekommen, wie ein Proof Assistent Tool die Qualität von Programmcode sicherstellt, müssen zunächst die Grundlagen dieser Sprache anhand von Beispielen erklärt werden. Anschließend wird näher auf das Zusammenspielen zwischen Programmcode und Proof Assistent eingegangen.\\
Es gibt bereits einige sehr erfolgreiche Forschungsprojekte, die Coq im Einsatz haben. Diese werden abschließend vorgestellt. Schlussendlich wird ein Fazit inklusive Ausblick in Hinsicht auf die Verwendbarkeit von Proof Assistent Tools gezogen.
\bigskip

\noindent
Schlagworte:
\begin{itemize}
	\item Proof Assistant
	\item Coq
	\item Programcodevalidierung
\end{itemize}


\section*{Leseanleitung}
Hinweise auf referenzierte Literatur und die daraus entnommenen Zitate, welche in eckigen Klammern angegebenen sind, werden im Literaturverzeichnis am Ende der Arbeit aufgeführt. Soll ein Begriff oder eine Formulierung besonders hervorgehoben werden, ist diese \textit{kursiv} geschrieben.
Abkürzungen werden bei erstmaligem Auftreten einmal in runden Klammern, anschließend an das Wort ausgeschrieben. Um den Lesefluss nicht zu stören, werden alle darauf folgenden Wiederholungen der Abkürzungen nicht immer explizit ausgeschrieben.
\\
Möglicherweise unbekannte Begriffe und Fachbegriffe werden bei ihrer ersten
Nennung \textbf{fett} gedruckt. Diese sind im Glossar in alphabetischer Reihenfolge aufgelistet und werden näher erklärt. Einzige Ausnahme hierbei sind Überschriften von Tabellen. Um ein zusammenhängendes Lesen der Arbeit zu erleichtern, werden bei Bedarf Erklärungen bereits im Text gegeben. Dabei wird davon ausgegangen, dass der Leser bereits mit grundlegenden Begriffen der Informatik vertraut ist. Ausgehend vom Wissensstand eines entsprechend vorgebildeten Lesers, werden demzufolge nur fachlich speziellere Begriffe erklärt.
\\
Um Unklarheiten zu vermeiden, werden Fachbegriffe in und zur Beschreibung von Bildern und Prozessen in ihrer originalen Sprache Englisch verwendet und nicht immer übersetzt.
\\
An den Stellen, an denen es der Ausführung des Textes dient, sind kurze Codebeispiele im Text eingebunden. Außerdem werden Abbildungen zur Veranschaulichung verwendet, um komplexe Prozesse einfacher und verständlicher zu machen. Größere Abbildungen befinden sich im Anhang und werden im passenden Textabschnitt referenziert. Damit soll der Lesefluss nicht gestört werden.
\section{Motivation}
Sample List
\begin{itemize}
	\item First item
	\item Second item
\end{itemize}

\section{Grundlagen}
\subsection{OpenHAB}
\subsubsection{Exemplarischer Beispielabschnitt:}
The open Home Automation Bus (openHAB) is an open source, technology agnostic home automation platform which runs as the center of your smart home!

Some of openHAB's strengths are:

Its ability to integrate a multitude of other devices and systems. openHAB includes other home automation systems, (smart) devices and other technologies into a single solution
To provide a uniform user interface and a common approach to automation rules across the entire system, regardless of the number of manufacturers and sub-systems involved
Giving you the most flexible tool available to make almost any home automation wish come true; if you can think it, odds are that you can implement it with openHAB.


\section{Beispielcode}
\begin{lstlisting}[language=coq,firstnumber=1,caption=Coq Beispiel,label=lst:api-config]
	Inductive day : Type :=
	| monday
	| tuesday
	| wednesday
	| thursday
	| friday
	| saturday
	| sunday.

	Definition next_weekday (d:day) : day :=
	match d with
	| monday => tuesday
	| tuesday => wednesday
	| wednesday => thursday
	| thursday => friday
	| friday => monday
	| saturday => monday
	| sunday => monday
	end.

	Compute (next_weekday friday).
	(* ==> monday : day *)
	Compute (next_weekday (next_weekday saturday)).
	(* ==> tuesday : day *)
\end{lstlisting}


\section{Vorlage-Einleitung}
\label{s:intro}

Hier kommt die Einleitung.


%%%%%%%%%%%%%%%%%%%%%%%%%%%%%%%%%%%%%%%%%%%%%%%%%%%%%%%%%%%%%%
\subsection{Ein Abschnitt der Einleitung}
\label{ss:intro:abc}

Einen Überblick findet man z.\,B.\ in \cite{Auer00:HTF}.

\begin{figure}[t]
	\centering
	
	\begin{subfigure}{0.45\linewidth}
		\centering
		\includegraphics[width=\linewidth]{\figdir/handorig}
		\caption{Originalbild}
		\label{FIG:arexorig}
	\end{subfigure}
	%
	\begin{subfigure}{0.45\linewidth}
		\centering
		\includegraphics[width=\linewidth]{\figdir/handaug}
		\caption{erweitertes Bild}
		\label{FIG:arexaugm}
	\end{subfigure}
	%
	\caption[AR Beispiel]
	{Beispiel eines Augmented Reality Systems: es folgt eine Beschreibung (Bilder aus \cite{Schmidt01:PAO})}
	\label{FIG:arex}
\end{figure}

Ein Beispiel wird in Abb.\ \ref{FIG:arex} gezeigt.
Das verwendete Objekt ist in Abb.\ \ref{FIG:arexorig} dargestellt, das Ergebnis in Abb.\ \ref{FIG:arexaugm}.

Eine Formel
\begin{equation}
\label{eq:cvp:test}
f(x) = \frac{1}{3} x + 5, \quad x \in \real.
\end{equation}

Und noch eine:
\begin{equation}
\label{eq:cvp:matvec}
\bm{M}  = \bm{Ax} \pi, \quad \bm{A} \in \real^{2 \times 2}, \bm{x} \in \real^2.
\end{equation}

Tabelle \ref{t:CodebookOverview} gibt einen Überblick über XYZ.

\begin{table}[t]
	\centering\small
	\input{\tabledir/CodebookOverview.tex}
	\caption[Testtabelle]{Datenselektion für verschiedene Testdatensätze.}
	\label{t:CodebookOverview}
\end{table}


%%% Local Variables: 
%%% mode: latex
%%% TeX-master: "thesis.tex"
%%% End: 

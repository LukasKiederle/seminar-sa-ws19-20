\section{Motivation}
Diese Ausarbeitung wird für das Fach Softwarearchitektur an der technischen Hochschule Rosenheim geschrieben.
Das Ziel ist es OpenHAB, ein Heimautomatisierungs-Tool, aus praktischer und technischer Sicht zu untersuchen. Dabei liegt der Fokus vor allem auf der Softwarearchitektur von OpenHAB. 

\section{Was ist OpenHAB}
OpenHAB ist eine technologie-unabhängige Open-Source-Automatisierungssoftware für Smart-Homes.
Sie wurde von Kai Kreuzer 2010 erstmals initiiert und wird mittlerweile durch die Community weiterentwickelt. OpenHAB ist in Java geschrieben und aktuell in der Version 2.4 erhältlich.\\
\\
Auf der offiziellen Website von OpenHAB \url{https://www.openhab.org/} sind drei klare Hauptziele definiert, die diese Software erreichen soll. Dabei ist ein Ziel die Plattformunabhängigkeit. Somit kann OpenHAB sowohl auf Linux, MacOS oder Windows betrieben werden. Auch das hosten mit Docker oder einem Raspberry Pi wird unterstützt.\\
Weiterhin soll es durch die Plugin-Architektur  möglich sein, fast jedes Gerät zu integrieren.
Es werden über 200 Technologien und mehrere tausende verschiedene Geräte unterstützt.\\
Das dritte Ziel weißt auf die vielen verschiedenen Automatisierungsmöglichkeiten hin, die OpenHAB zu bieten hat. Dabei werden Auslöser, Aktionen, Skripte und auch Voice-Kontrolle genannt.

\section{Bewertung OpenHAB}


\section{Datenintegriertät und Sicherheit}
\url{https://www.openhab.org/docs/installation/security.html}
\begin{itemize}
	\item Through the command line console, which is done through SSH and thus always authenticated and encrypted. You will find all details about this in the Console documentation.
	\item Through HTTP(S) over \url{https://<ip>:8443}
	\item Options for Secure Remote Access
	\begin{itemize}
		\item VPN: The most secure option is probably to create a VPN connection to your home network
		\item myopenHAB Cloud Service with a tunnel that forwards all requests to the openHAB instance
		\item Running openHAB Behind a Reverse Proxy: A reverse proxy simply directs client requests to the appropriate server. This means you can proxy connections to \url{http://mydomain_or_myip} to your openHAB runtime.
	\end{itemize}
\end{itemize}



\section{OpenHAB aus technischer Sicht}
In diesem Kapitel sind die grundlegenden Komponenten, die OpenHAB verwendet, tabellarisch dargestellt. Anschließend wird detaillierter auf die einzelnen Elemente eingegangen.

\begin{longtable}{| p{2cm} | p{13cm}|}
	\hline
	\textbf{Konzept} & \textbf{Beschreibung} \\
	\hline \hline
	\centering Bindings & sind die openHAB-Komponenten, die die Schnittstelle zur elektronischen Interaktion mit Geräten bereitstellt.  \\
	\hline
	\centering Things & sind die erste von openHAB (Software) generierte Darstellung von Geräten. \\
	\hline
	\centering Channels & sind die openHAB (Software)-Verbindung zwischen "`Dingen"' und "`Gegenständen"'. \\
	\hline
	\centering Items & sind die von openHAB (Software) generierte Darstellung von Informationen über die Geräte.\\
	\hline
	\centering Rules & führen automatische Aktionen durch (in einfachster Form: wenn "`dies"' passiert, wird openHAB "`das"' tun).\\
	\hline
	\centering Sitemaps & ist die von openHAB (Software) generierte Benutzeroberfläche (Website), die Informationen präsentiert und Interaktionen ermöglicht.\\
	\hline
	\caption{OpenHAB Komponenten}
	\label{table:openhub-components}
\end{longtable}

\subsection{Bindings}
\begin{itemize}
	\item Typische Bindings
	\item Screenshot?
	\item Geräteerkennung
\end{itemize}

\subsection{Things}
\begin{itemize}
	\item 
	\item 
\end{itemize}

\subsection{Channels}
\begin{itemize}
	\item  Dient als Kommunikationsweg für von Openhab zu things.
	\item Channel sind beispielsweise bei einer Lampe die Brightness.
\end{itemize}

\subsection{Items}
\begin{itemize}
	\item Item bedeutet nicht Gerät oder Service
	\item Items können gruppiert werden
	\item Ein Item stellt einer der Basistypen dar: String, Number, Color oder Group.
	\item Items werden über mit Hilfe von Bindings mit der Außenwelt verbunden
	\item \url{https://www.openhab.org/docs/configuration/items.html#items}
\end{itemize}
\begin{lstlisting}[language=java,firstnumber=1,caption=Item-Gruppierung Beispiel,label=lst:group-items]
	Group groundFloor
	Switch kitchenLight (groundFloor)
	Switch livingroomLight (groundFloor)
\end{lstlisting}

\subsection{Rules}
\begin{itemize}
	\item Rules stelln wenn dann Beziehungen dar
	\item Diese können sowohl über zusammenklicken, als auch programmatisch erstellt werden. 
	\item Das Zusammenklicken basiert auf einem noch nicht fertigen Feature namens experimental rules. Zum Zeitpunkt dieser Arbeit können dadurch schon einige Standardrules definiert werden. Allerdings fehlt beispielsweise noch die Vergleichsoption größer oder größer-gleich
	\item Codeblock \ref{lst:sample-rule} zeigt ein programmatisch erstellte Rule.
	\begin{itemize}
		\item Eine Rule besteht immer aus einem Namen, einer when-Bedinung und einem then Abschnitt.
		\item Name dient als Zuordnung
		\item When ist der Trigger bzgw. Auslöser der Aktion, welche im then Block definiert ist.
		\item Diese Rule prüft, ob die Lautstärke des Items (TV) Dominik\_volumen sich verändert hat.
		Wenn das der Fall ist, wird eine geprüft, wiehoch denn die aktuelle Lautstärke ist.
		Folglich wird bei unter 20 die Lampe gedimmt und bei über 20 die Lampe erhellt.
		\item Falls etwas nicht klappen sollte, können auch Debug-Ausgaben mit dem Kommando logDebug geschrieben werden.
	\end{itemize}
\end{itemize}
\begin{lstlisting}[language=java,firstnumber=1,caption=Beispiele Rule Beispiel,label=lst:sample-rule]
rule "React on Volume (LGWebOSTVUH620VDominik_Volume) change"
when
	Item LGWebOSTVUH620VDominik_Volume changed
then
	logDebug("React some changes on Volume", "some Message" + LGWebOSTVUH620VDominik_Volume.state.toString)
if ( LGWebOSTVUH620VDominik_Volume.state >= 20 ) {
	HueWhiteLamp2_Brightness.sendCommand(80)
}
else {
	HueWhiteLamp2_Brightness.sendCommand(5)
}
end
\end{lstlisting}

\subsection{Sitemaps}

\subsection{Api}
\url{https://www.openhab.org/docs/configuration/restdocs.html}
\begin{itemize}
	\item Item ein-/ausschalten
	\item Eine List von allen Items, Sitemaps ausgeben lassen
	\item Mit einem Editor auf die ganzen Komponenten zugreifen:
	\begin{itemize}
		\item Visual Studio Code installieren
		\item Openhab Extension installieren
		\item Geteiltes Openhab Laufwerk als Ordner öffnen
		\item Openhab Extension konfigurieren
		\item Es werden auch andere Editoren unterstützt
	\end{itemize}
\end{itemize}

\section{Verwendung von OpenHAB}
\subsection{Integration der Big Player}
\begin{itemize}
	\item Amazon Alexa und Echo Dot Integration möglich
	\begin{itemize}
		\item Alexa:
		\begin{itemize}
			\item This certified Amazon Smart Home Skill allows users to control their openHAB powered smart home with natural voice commands. Lights, locks, thermostats, AV devices, sensors and many other device types can be controlled through a user's Alexa powered device like the Echo or Dot.
			\item \url{https://www.openhab.org/docs/ecosystem/alexa/}
			\item \url{https://www.openhab.org/addons/bindings/amazonechocontrol/}
		\end{itemize} 
		\item Google Home
		\begin{itemize}
			\item Google Home Integration möglich
			\item With the Action you can voice control your openHAB items and it supports lights, plugs, switches and thermostats. The openHAB Action comes with multiple language support like English, German or French language.
			\item The openHAB Action links your openHAB setup through the myopenHAB.org cloud service to the Google Assistant platform
			\item openHAB Cloud Connector configured using myopenHAB.org . (Items DO NOT need to be exposed to and will not show up on myopenHAB.org
			, this is only needed for the IFTTT service!)
			Google account.
			Google Home or Google Home mini.
			
			\url{https://www.openhab.org/docs/ecosystem/google-assistant/}
		\end{itemize} 
	\end{itemize}
	
\end{itemize}

\subsection{Beispiel Aufbau eines OpenHAB Smart-Homes}
\begin{itemize}
	\item OpenHAB auf Raspberry Pi 3/4 installiert
	\item Welche Geräte haben wir mit OpenHAB verbunden?
	\begin{itemize}
		\item Spotify
		\begin{itemize}
			\item Lautstärkeregler
			\item Aktueller Song Display
		\end{itemize}
		\item LG Smart TV
		\begin{itemize}
			\item Lautstärkeregler
			\item An- und ausschalten
			\item One-Way-Chat
		\end{itemize}
		\item Lampen
	\end{itemize}
	\item Wie haben wir die Geräte verbunden?
	\begin{itemize}
		\item \textbf{Verschiedene Binding:}
		\item Spotify Binding
		\item LG Smart TV Binding
	\end{itemize}
	\item On the server the configuration is stored somewhere in userdata (/var/lib/openhab2 for apt-get installs).
	In an upgrade the userdata folder is preserved when using apt-get.
\end{itemize}
\begin{minipage}{\textwidth}
	\centering
	\captionsetup{type=figure}
	\includegraphics[width=1\textwidth]{\figdir/verteilungsarchitektur.png}
	\caption{Verteilungsarchitektur \label{fig:verteilungs-architektur}}
\end{minipage}
\begin{minipage}{\textwidth}
	\centering
	\captionsetup{type=figure}
	\includegraphics[width=1\textwidth]{\figdir/activitydiagram-lg-light-rule.png}
	\caption{Aktivitätsdiagram für eine Rule \label{fig:activity-diagram}}
\end{minipage}

\subsection{Umgang mit OpenHAB}
\begin{itemize}
	\item Das meiste klickt mans ich zusammen: Bindings, Rules, Channels, Items, Things
	\item Implementierung von rules scheint idiotensicher, weil:
	\begin{itemize}
		\item einfacher Syntax
		\item Abhängigkeiten managed Openhab
	\end{itemize}
	\item Bindings schreiben scheint eher schwieriger
\end{itemize}

\section{Fazit}
\subsection{Stärken}
Some of openHAB's strengths are:

Its ability to integrate a multitude of other devices and systems. openHAB includes other home automation systems, (smart) devices and other technologies into a single solution
To provide a uniform user interface and a common approach to automation rules across the entire system, regardless of the number of manufacturers and sub-systems involved
Giving you the most flexible tool available to make almost any home automation wish come true; if you can think it, odds are that you can implement it with openHAB.
\subsection{Schwächen}
\textbf{Wollen wir das hier als SWOT Analyse aufziehen?}
\begin{itemize}
	\item Integration von USB-Geräten scheint eher kompliziert. Vor allem auf Raspberry Pi
	\item Serial Binding wird nicht angezeigt
	\begin{itemize}
		\item Mikrofon an Raspberry Pi oder anderes Geräte verbinden
		\item Input des Mikrofons über OpenHAB an ein Ausgabegerät, wie zum Beispiel eine Bluetooth Box, senden und abspielen
		\item Raspberry hat da auch für große Probleme bei der Geräteerkennung gesorgt - USB gerät wurde nicht im devices Verzeichnis aufgeführt und somit konnte auch keine Verbindung mit OpenHAB aufgebaut werden
		\item OpenHAB Serial Device Binding wurde auch nicht angezeigt, um Geräte darüber zu suchen
	\end{itemize}
\end{itemize}

\section{Infos:}
\textbf{Ausgangslage}
Untersuchen Sie die Architektur und Features von OpenHAB und
schreiben Sie ein Beispielanwendung.
Mit myOpenHub existiert eine kostenlose Plattform die sie nutzen
können.

\textbf{Beantworten Sie dabei}
\begin{itemize}
 \item Aktueller Status des Projekts und  
 \item Integration der Big Player wie Alexa und Google Home
 \item Welche Tools und Konzepte und APIs gibt es
 \item Welche Deployment Modi und Betriebsmodi existieren
 \item Untersuchen Sie auch Aspekte wie Datenintegriertät und Sicherheit
\end{itemize}

\textbf{Unterlagen Linkes}
\begin{itemize}
	\item \url{https://www.myopenhab.org/}
	\item \url{https://www.openhab.org/}
	\item \url{https://jaxenter.de/openhab-2-4-78711}
\end{itemize}


\section{Vorlage mit Samples}

Einen Überblick findet man z.\,B.\ in \cite{Auer00:HTF}.

\begin{figure}[t]
	\centering
	
	\begin{subfigure}{0.45\linewidth}
		\centering
		\includegraphics[width=\linewidth]{\figdir/handorig}
		\caption{Originalbild}
		\label{FIG:arexorig}
	\end{subfigure}
	%
	\begin{subfigure}{0.45\linewidth}
		\centering
		\includegraphics[width=\linewidth]{\figdir/handaug}
		\caption{erweitertes Bild}
		\label{FIG:arexaugm}
	\end{subfigure}
	%
	\caption[AR Beispiel]
	{Beispiel eines Augmented Reality Systems: es folgt eine Beschreibung (Bilder aus \cite{Schmidt01:PAO})}
	\label{FIG:arex}
\end{figure}

Ein Beispiel wird in Abb.\ \ref{FIG:arex} gezeigt.
Das verwendete Objekt ist in Abb.\ \ref{FIG:arexorig} dargestellt, das Ergebnis in Abb.\ \ref{FIG:arexaugm}.

Eine Formel
\begin{equation}
\label{eq:cvp:test}
f(x) = \frac{1}{3} x + 5, \quad x \in \real.
\end{equation}

Und noch eine:
\begin{equation}
\label{eq:cvp:matvec}
\bm{M}  = \bm{Ax} \pi, \quad \bm{A} \in \real^{2 \times 2}, \bm{x} \in \real^2.
\end{equation}

Tabelle \ref{t:CodebookOverview} gibt einen Überblick über XYZ.

\begin{table}[t]
	\centering\small
	\input{\tabledir/CodebookOverview.tex}
	\caption[Testtabelle]{Datenselektion für verschiedene Testdatensätze.}
	\label{t:CodebookOverview}
\end{table}


%%% Local Variables: 
%%% mode: latex
%%% TeX-master: "thesis.tex"
%%% End: 
